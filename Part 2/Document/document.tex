\documentclass[a4paper]{article}
\usepackage{graphicx}
\usepackage{float}
\author{Hein van Beers \qquad Student number: 0765658 \\{\tt h.a.v.beers@student.tue.nl}\\ \\ Jeroen van Hoof \qquad Student number: 0778486 \\{\tt j.m.a.p.v.hoof@student.tue.nl}}
\title{Automated Reasoning\\
	 \large Practical assignment, Part II}
\begin{document}
	\maketitle
	
	\section*{Problem 1: Non-self-supporting Villages}
	Three non-self-supporting villages A, B and C in the middle of nowhere consume one food package each per time unit. The required food packages are delivered by a truck, having a capacity of 300 food packages. The locations of the villages are given, together with the number of time units it takes the truck to travel from one village to another, including loading or delivering. The truck has to pick up its food packages at location S containing an unbounded supply. The villages only have a limited capacity to store food packages: for A and B this capacity is 120, for C it is 200.
	
	\subsection*{Solution:}
	We generalize this problem to $n$ villages and a truck capacity of $C$. We also have a variable $m$ which represents the amount of steps we perform in the model. Each step corresponds to a change in the position of the truck. For each village we then introduce $n * (m+2)$ integer variables $v_{i\_j}$ for $i=1,...,n$ and $j=0,...,m+1$, where $i$ indicates the number of each village and $j$ is the number of steps that have been performed. So now we have a variable $v_{i\_j}$ that represents the amount of food packages in village $i$ after $j$ steps have been performed. Next, we also introduce $m+2$ integer variables $p_j$, $t_j$ and $d_j$ for $j=0,...,m+1$, where $p_j$ represents the position of the truck after $j$ number of steps have been performed, $t_j$ represents the load of the truck after $j$ number of steps have been performed, and $d_j$ represents the amount of food packages that are being delivered at step $j$.\\
	
	Since for each village $i$ the food capacity $v_{i\_j}$ may not exceed its maximum capacity $max_i$, neither may get below zero for any step $j$, we introduce the following formula
\[ \bigwedge_{j=0}^{m+1} \bigwedge_{i=1}^n (v_{i\_j} >= 0) \wedge (v_{i\_j} <= max_i).\]
Where $max_i$ is the maximum capacity of village $i$. In the table below you can see the maximum capacities of villages A, B and C, together with their corresponding integer values.

\begin{table}[H]
\centering
\caption{Maximum capacities of each village}
\label{my-label}
\begin{tabular}{c|c|c}
\textbf{Village} & \textbf{Corresponding $i$} & \textbf{Maximum capacity} \\ \hline
A	&	1	&	120\\ \hline
B	&	2	&	120\\ \hline
C	&	3	&	200
\end{tabular}
\end{table}

	The position $p_j$ of the truck at each step $j$ should be bounded below by zero and above by the number of villages we have in total. Also the capacity of the truck $t_j$ at each step $j$ should be bounded below by zero and above by the maximum capacity of the truck $C$. Finally, we also need a bound on the capacity that is being delivered $d_j$ at each step $j$, since we can never deliver a negative value. The bounds for these three variables can be represented with the following formula
	\[ \bigwedge_{j=0}^{m+1} (p_j >= 0) \wedge (p_j <= n) \wedge (t_j >= 0) \wedge (t_j <= C) \wedge (d_j >= 0).\]
	Where $m$ is the amount of steps we take, $n$ is the total number of villages (in our case 3), and $C$ is the maximum capacity of the truck.\\

Finally, we need to implement the steps that can be taken at each step of the model. The steps that can be taken depend on the current position of the truck, since in the next step, the truck can only go to neighbouring villages. So let $X_i$ be the set of neighbouring villages of village $i$ and let $f_i(x)$ be a function that determines the costs of going from village $i$ to village $x$. We now can easily express the steps and their corresponding changes in the variables. The formula below represents the possibilities for step $j$ and current position $0$

\[ (p_j = 0) \Rightarrow ( (d_j = 0) \wedge \]
\[ (\bigvee_{x \in X_0} ((p_{j+1} = x) \wedge \]
\[ (\bigwedge_{1 \leq y \leq n} (v_{y\_j+1} = v_{y\_j} - f_0(y))) \wedge \]
\[ (t_{j+1} = t_j - d_{j+1})))).\]

Similarly, we can also do this for all other positions $i$, but then we only need to add that the delivery changes and that the capacity of the villages also changes if the truck delivers to its village. We can see the possibilities for step $j$ and current position $i \neq 0$ in the formula below

\[ (p_j = i) \Rightarrow ( (d_j \leq max(i) - v_{i\_j}) \wedge \]
\[ (\bigvee_{x \in X_i} ((p_{j+1} = x) \wedge \]
\[ (\bigwedge_{1 \leq y \leq n} (((y=i) \wedge (v_{y\_j+1} = v_{y\_j} - f_i(y) + d_j)) \vee \]
\[ ((y \neq i) \wedge (v_{y\_j+1} = v_{y\_j} - f_i(y))))) \wedge \]
\[ (t_{j+1} = t_j - d_{j+1})))).\]


Next, we express the condition that the number of delivered packages $d$ in village $i$ is limited by the capacity of this village. Also, when the truck is in position zero (Supplier) it should not deliver any package.

\[\bigwedge_{j=0}^m \bigwedge_{i=1}^n (p_j = i) \rightarrow (d_j \leq (max(i) - a_{ij})) \;\;\wedge\]
\[\bigwedge_{j=0}^m (p_j = 0) \rightarrow (d_j = 0) .\]

Finally, when the truck is in a given position, the next position should be a neighboring village. Also, when moving to another position the amount of food in every village will decrement depending on the travel time. In order to express this conditions, we introduce $\textbf{C}_i$ as the sets of neighbors of village $i$, and the mappings $f_i:\textbf{C}_i \rightarrow \textbf{N}$ to determine the time required to travel from $i$ to one of its neighbor villages, eg. lets say that the set of neighbors of village $1$ is $\textbf{C}_1 = \{0,2\}$ then villages 0 and 2 are connected to village 1, and $f_1(0)$ is the time required to go from village 1 to village 0.

Using all this elements, the formula that expresses this condition is the following:

\[\bigwedge_{j=0}^{m-1} \bigwedge_{i=0}^n (p_j = i) \rightarrow (\bigvee_{k \in \textbf{C}_i} (p_{j+1}=k \;\;\wedge\ a_{ij+1} = a_{ij} + d_{j} - f_i(k) \;\;\wedge\]
\[\qquad \qquad \qquad \bigwedge_{1\leq l \leq n:i\neq l} a_{lj+1} = a_{lj} - f_i(k) \;\;\wedge\]
\[\qquad \qquad(k=0) \rightarrow (t_{j+1} \geq t_j) \;\;\wedge\]
\[\qquad \qquad \qquad \; \;(k\neq0) \rightarrow (t_{j+1} = t_j - d_{j+1}) )).\]


\begin{table}[H]
\centering
\caption{Demand of each type of pallets}
\label{my-label}
\begin{tabular}{c|c|c}
\textbf{Pallet type} & \textbf{Corresponding k} & \textbf{Demand (pallets)} \\ \hline
Nuzzles              & 1                        & 4                  		\\ \hline
Prittles             & 2                        & ?                  		\\ \hline
Skipples             & 3                        & 8                 		\\ \hline
Crottles             & 4                        & 10                 		\\ \hline
Dupples              & 5                        & 5                 
\end{tabular}
\end{table}

For getting the answer to the maximization problem, we just try different values for the amount of pallets of prittles that need to be delivered and see if we get a solution. When we find a solution for a value $x$ for the amount of prittles that we deliver and we cannot find a solution for value $x+1$, we know that $x$ is the maximum number of pallets of prittles that can be delivered.\\

After trying several values, we found that 18 pallets of prittles is the maximum amount we can deliver, since the program does not yield a solution for 19 pallets of prittles.\\

The total formula now consists of the conjunction of all these ingredients, that is,
\[ \bigwedge_{i=1}^n (\sum_{k=1}^5 (\sum_{j=1}^8 (p_{ijk}*w_k)) \leq 7800) \;\; \wedge \]
\[ \bigwedge_{i=1}^n (\neg (\bigvee_{j=1}^8 p_{ij2}) \vee \neg (\bigvee_{m=1}^8 p_{im4})) \;\; \wedge \]
\[ \bigwedge_{i=3}^n \bigwedge_{j=1}^8 (\neg p_{ij3}) \;\; \wedge \]
\[ \bigwedge_{i=1}^n \bigwedge_{j,m:1 \leq j < m \leq 8} (\neg p_{ij5} \vee \neg p_{im5}) \;\; \wedge \]
\[ \bigwedge_{i=1}^n \bigwedge_{j=1}^8 \bigwedge_{k,m:1 \leq k < m \leq 5} (\neg p_{ijk} \vee \neg p_{ijm}) \;\; \wedge \]
\[ \bigwedge_{k=1}^5 ( \sum_{i=1}^n ( \sum_{j=1}^8 p_{ijk} ) = a_k) \]


This formula is easily expressed in SMT syntax, for instance, for $n=6$ one can generate

{\footnotesize

{\tt (benchmark Assignment1.smt}

{\tt :logic QF\_LIA}

{\tt :extrapreds (}

{\tt ;Truck 1 }

{\tt (p111) (p112) (p113) (p114) (p115) }

{\tt (p121) (p122) (p123) (p124) (p125) }

{\tt (p131) (p132) (p133) (p134) (p135) }

{\tt (p141) (p142) (p143) (p144) (p145) }

{\tt (p151) (p152) (p153) (p154) (p155) }

{\tt (p161) (p162) (p163) (p164) (p165) }

{\tt (p171) (p172) (p173) (p174) (p175) }

{\tt (p181) (p182) (p183) (p184) (p185) }

{\tt ;Truck 2 }

{\tt (p211) (p212) (p213) (p214) (p215) }

{\tt (p221) (p222) (p223) (p224) (p225) }

$\cdots \cdots$

{\tt )}

{\tt :formula}

{\tt   (and}

{\tt (<= (+ }

{\tt (* (ite p111 1 0) 700) $\cdots$ (* (ite p181 1 0) 700)}

{\tt (* (ite p112 1 0) 800) $\cdots$ (* (ite p182 1 0) 800)}

$\cdots \cdots$

{\tt (* (ite p115 1 0) 100) $\cdots$ (* (ite p185 1 0) 100)}

{\tt ) 7800)}

$\cdots \cdots$

{\tt (or }

{\tt (not (or p112 p122 p132 p142 p152 p162 p172 p182)) }

{\tt (not (or p114 p124 p134 p144 p154 p164 p174 p184)) }

{\tt ) }

$\cdots \cdots$

{\tt )) }
}

	\subsection*{Problem 1a}
	We need to show that it is impossible to deliver food packages in such a way that each of the villages consumes one food package per time unit forever.
	
	\subsection*{Problem 1b}
	For this question we need to show that with a truck capacity of 320 food packages, it is possible to deliver food packages in such a way that the villages consume each one food package per time unit forever.

	\subsection*{Problem 1c}
	Finally, we also need to check the same as for question 1b, but now with a truck capacity of 318 food packages.

Applying {\tt yices -m -smt Assignment1.smt} yields the following result
within a fraction of a second:

{\footnotesize

{\tt sat }

{\tt (= p185 false)}

{\tt (= p445 false)}

{\tt (= p213 true)}

{\tt (= p214 false)}

{\tt (= p452 false)}

{\tt (= p233 true)}

{\tt (= p435 false)}

{\tt (= p474 true)}

{\tt (= p623 false)}

{\tt (= p531 true)}

{\tt (= p662 false)}

$\cdots \cdots$ }

The values that are are true for each truck are 
\[p113, p125, p134, p143, p153, p163, p173, p183 \]
\[p213, p222, p233, p242, p255, p262, p272, p282 \]
\[p312, p322, p332, p342, p352, p362, p372, p382 \]
\[p414, p424, p444, p464, p474, p485 \]
\[p512, p525, p531, p541, p552, p562, p572, p582 \]
\[p611, p621, p634, p644, p665, p674, p684 \]
Expressed in a picture this yields



We check that indeed all requirements are satisfied and we also see that the amount of pallets of prittles we can deliver is 18, since the program does not yield a solution for 19 pallets of prittles.\\

{\bf Remark:}


{\bf Generalization:} 


	\section*{Problem 2: Placing components on a power grid}
	

	\subsection*{Solution:}
	
	
	\section*{Problem 3: Ordering jobs}



	\subsection*{Solution:}



{\bf Remark:}


{\bf Generalization:} 


\end{document}